%%%%%%%%%%%%%%%%%%%%%%%%%%%%%%%%%%%%%%%%%%%%%%%%%%
\section{Expected outcomes of the project}
\notePEPR{Décrire les résultats scientifiques attendus, ainsi que leurs impacts potentiels dans le domaine du PEPR. }
\notePEPR{Décrire comment les résultats scientifiques s’inscrivent dans une trajectoire de transformation globale socio-économique-technique-écologique-territorial ; en quoi ces résultats scientifiques sont-ils transformants et transposables pour assurer les transformations nécessaires des Villes et bâtiments en villes et bâtiments durables.}
\notePEPR{Détailler la stratégie de diffusion et de valorisation des résultats, y compris les actions de promotion de la culture scientifique. Décrire comment le projet envisage de promouvoir les résultats de la recherche auprès des autorités publiques concernées (État, collectivités, agences) et auprès du grand public.}
\notePEPR{Préciser finement comment le projet nourrira le reste du PEPR et le reste de la stratégie nationale.}
\notePEPR{Décrire finement comment les résultats du projet pourront être réutilisés par la communauté (codes, connaissances, données, préconisations, retours d’expériences, ...) permettant de proposer des solutions pour une Ville Durable et Bâtiment Innovant.}
\notePEPR{Expliquer la stratégie de préparation et de diffusion des résultats et données liées au projet.}
\notePEPR{Préciser les retombées à court terme (pour des expérimentations lancées pendant le projet) ou à moyen terme (pour des expérimentations issues du projet).}


\section{Funding justification}
\notePEPR{Indiquer les détails des moyens matériels, financiers et humains, qui seront mobilisés dans le cadre du projet et leur adéquation par rapport aux objectifs. L’établissement coordinateur justifiera les moyens qu’il demande au titre de l’ensemble du consortium, sur la durée du projet, en indiquant les mécanismes qu’il mettra en œuvre pour distribuer ces moyens ainsi que ceux mis à disposition par les partenaires du projet, ou ceux qui seront obtenus en cofinancement. Les dépenses éligibles (décrites dans le Règlement Financier) doivent avoir un lien direct avec les actions de recherche du projet.}

\begin{comment}

% exemple de table reprise du template unofficial ANR
\begin{table}[h]
	\centering
    \small
	\begin{tabular}{p{0.31\textwidth} r r r r r}
		\hline
        \rowcolor{headcolor!20}
	    & \textbf{Partner} & \textbf{Partner} & \textbf{Partner} & \textbf{Partner} & \textbf{Partner} \\
        \rowcolor{headcolor!20}
	    & \textbf{\colUE{LITA (UE)}} & \textbf{\colEGS{LMQ (EGS)}} & \textbf{\newline\colINR{LMD (INR)}} & \textbf{\newline\colUT{CGCS (UT)}} & \textbf{\newline\colTUT{FI (TUT)}} \\
		\hline
		Staff expenses & 250~000\euro & 100~000\euro & 200~000\euro & 75~000\euro & 100~000\euro \\
		Instruments and material costs & 5~000\euro & 7~500\euro & 3~000\euro & 9~000\euro & 6~000\euro \\
		Building and ground costs & 3~000\euro & 98~000\euro & 0\euro & 0\euro & 0\euro \\
		Outsourcing / subcontracting & 0\euro & 0\euro & 0\euro & 11~000\euro & 7~000\euro \\
		Overheads costs\footnote{including missions expenses, general and administrative costs \& other operating expenses} & 8~000\euro & 6~500\euro & 12~345\euro & 5~000\euro & 11~000\euro \\
		Administrative management \& structure costs & 100\euro & 10\euro & 150\euro & 30\euro & 2\euro \\
		\hline
        \rowcolor{headcolor!20}
		\textbf{Sub-total} & \textbf{266~100\euro} & \textbf{212~010\euro} & \textbf{215~495\euro} & \textbf{100~030\euro} & \textbf{124~002\euro} \\
		\hline
        \rowcolor{headcolor!40}
		\textbf{Requested} & \multicolumn{5}{c}{\textbf{917~637\euro}} \\
		\hline
	\end{tabular}
	\caption{Requested means by item of expenditure and by partner.}
\end{table}

\end{comment}

