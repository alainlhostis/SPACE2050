%%%%%%%%%%%%%%%%%%%%%%%%%%%%%%%%%%%%%%%%%%%%%%%%%%
\section{Project organisation and management}

%%%%%%%%%%%%%%%%%%%%%%%%
\subsection{Project manager}
\notePEPR{Fournir les éléments permettant d’évaluer la capacité du Responsable du projet à piloter et coordonner le projet1. CV du responsable du projet (section 8) : il est vivement conseillé aux candidats de rédiger leur CV en anglais et de ne pas utiliser d'abréviations car les évaluations peuvent être effectuées par des non francophones.}

\subsection{Organization of the partnership}
\notePEPR{Présenter les éléments permettant d’apprécier la qualité du groupement. Décrire les compétences et la qualité des entités mobilisées dans le cadre du projet. Les éléments d’appréciation du groupement peuvent être des réalisations passées, des indicateurs (publications), l’adéquation de l’engagement scientifique de l’entité par rapport aux objectifs du PEPR, les infrastructures dont elle dispose ou auxquelles elle a accès, etc.}
\notePEPR{Présenter les postes clés du projet, par exemple les responsables d’un lot. La granularité du détail devra être ajustée à l’ampleur du projet. Typiquement, 3 à 6 personnes peuvent être décrites. Une liste des autres acteurs importants peut être donnée.}
\notePEPR{Préciser pour chacun des partenaires (parties prenantes), sa contribution, son apport autour de ses métiers et ses pratiques (compléter le tableau « métiers et pratiques » – annexe partenariats).}
\notePEPR{Précisez pour les laboratoires non financés (Européens, Internationaux), leur contribution et leur apport dans le projet ainsi que leur rôle dans la réalisation des objectifs (compléter le tableau « laboratoires non financés » – annexe partenariats).}
\notePEPR{Présenter les terrains envisagés afin d’ancrer les recherches proposées sur un territoire (compléter le tableau « terrains envisagés » – annexe partenariats).}
\notePEPR{Montrer la complémentarité des partenaires du consortium au regard des objectifs du projet et la valeur ajoutée apportée par chaque entité.}

\subsection{Management framework}
\notePEPR{Préciser l’organisation entre partenaires, ainsi que les modalités de pilotage du projet.}
\notePEPR{Définir la méthode de suivi des indicateurs et des jalons. Décrire la façon dont seront gérés les différents risques anticipés pour le projet.}
\notePEPR{Préciser les modalités d’accès aux ressources partagées, de valorisation des résultats, et de partage de la propriété intellectuelle et industrielle.}
\notePEPR{Indiquer précisément le lien avec le pilotage global du PEPR et l’intégration dans ses outils de suivi.}
\notePEPR{Indiquer précisément le lien avec les centres opérationnels et les mises en synergie proposées ; expliciter, si c’est nécessaire les complémentarités avec d’autres projets déposés dans le même appel ou dans d’autres appels.}

\subsection{Institutional strategy}
\notePEPR{Décrire comment le projet s’inscrit dans la stratégie des établissements partenaires et comment les établissements soutiennent le projet dans sa durée. Décrire quels sont les principaux moyens et dispositifs mis à disposition par les établissements au projet.}
\notePEPR{Indiquez comment le projet participe à l’animation de la communauté VDBI à une échelle nationale en impliquant les établissements participants, laboratoires et partenaires.}




\begin{table}[h]
	\centering
    \small
    \begin{tabular}{p{0.10\textwidth} p{0.13\textwidth} p{0.24\linewidth} p{0.14\linewidth} p{0.11\linewidth} p{0.12\linewidth}}
		\hline
        \rowcolor{headcolor!20}
	    \textbf{Researcher} & \textbf{Person.month} & \textbf{Call, funding agency, \newline grant allocated} & \textbf{Project's title} & \textbf{Scientific \newline coordinator} & \textbf{Start--End} \\
		\hline
		... & ... & ... & ... & ... & ... \\
		... & ... & ... & ... & ... & ... \\
		\hline
	\end{tabular}
	\caption{Implication of the scientific coordinator and partner's scientific leader in on-going project(s)}
\end{table}



\begin{table}[h]
	\centering
    \small
    \begin{tabular}{p{0.10\textwidth} p{0.13\textwidth} p{0.24\linewidth} p{0.14\linewidth} p{0.11\linewidth} p{0.12\linewidth}}
		\hline
        \rowcolor{headcolor!20}
	    \textbf{Researcher} & \textbf{Person.month} & \textbf{Call, funding agency, \newline grant allocated} & \textbf{Project's title} & \textbf{Scientific \newline coordinator} & \textbf{Start--End} \\
		\hline
		... & ... & ... & ... & ... & ... \\
		... & ... & ... & ... & ... & ... \\
		\hline
	\end{tabular}
	\caption{Implication of the scientific coordinator in on-going project(s)}
\end{table}

The project \textsc{MyProJecT} is coordinated by \colUE{Stéphanie MARTIN (LITA)}. It is organized in four technical WPs divided in tasks. Each WP is under the responsibility of a well-identified coordinator namely \colUE{Stéphanie MARTIN (LITA)} for WP~\ref{wp:ProjMgt}, \colEGS{Jing ZANG (LMQ)} for WP~\ref{wp:Data}, \colINR{Assa DIALA (LMD)} for WP~\ref{wp:Classif}, and \colINR{Richard MEUNIER (LMD)} for WP~\ref{wp:Exp}.

%%%
\paragraph*{\colUE{S. MARTIN}} \hspace{-1em} is blablabla at the \colUE{Laboratoire d'Informatique Théorique et Appliquée} with Université de l'excellence. \lipsum[7] 

Beside the coordination (WP~\ref{wp:ProjMgt}), she will be involved in WP~\ref{wp:Data}, \ref{wp:Classif} and \ref{wp:Exp}.

%%%
\paragraph*{\colEGS{J. ZANG}} \hspace{-1em} is the leader of WP 1. He is bla bla bla at the \colEGS{Laboratoire de Métaphysique Quantique} with École générale supérieure. \lipsum[2] 

He will participate in all tasks of WP~\ref{wp:Data}.

%%%
\paragraph*{\colINR{A. DIALA}} \hspace{-1em} is the leader of WP~\ref{wp:Classif}. She is bla bla bla at the \colINR{Laboratoire de Mathématiques très Discrètes} with Institut national de la recherche. \lipsum[8] 

He will work on several tasks of WP~\ref{wp:Classif}.

%%%
\paragraph*{\colINR{R. MEUNIER}} \hspace{-1em} is the leader of WP~\ref{wp:Exp}. He is bla bla bla at the \colINR{Laboratoire de Mathématiques très Discrètes} with Institut national de la recherche. \lipsum[9]

He will work on WP~\ref{wp:Classif} and \ref{wp:Exp}.

%%%
\paragraph*{Other project members:} \colEGS{A.~MANSOURI} (Professor at \colEGS{Laboratoire de Métaphysique Quantique} with École générale supérieure), he has a strong expertise in bla bla bla bla. He will be involved in Tasks~\ref{tsk:ExpYT} and \ref{tsk:ExpTT}.

\colUE{R. GONZALES} (Associate Professor at the \colUE{Laboratoire d'Informatique Théorique et Appliquée} with the Université de l'excellence), he is specialized in bla bla bla. He will participate in Tasks~\ref{tsk:ExpYT} and \ref{tsk:ExpTT}.

\colUT{D. WAKEFIELD} (Assistant Professor at the \colUT{Center for Great Computer Science} with the University of Tadborough), a specialist of blablabla bla blabla blabla. She will take part in Tasks~\ref{tsk:DataOT} and \ref{tsk:DataLT}.
.

\colTUT{O. MÜLLER} (Professor at the \colTUT{Forschungszentrum für Informatik} with Technische Universität Turmstadt), he works on bla bla bla bla. He will participate in Tasks~\ref{tsk:ExpYT} and \ref{tsk:ExpTT}.




%%%%%%%%%%%%%%%%%%%%%%%%





